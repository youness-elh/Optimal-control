\documentclass[xcolor=dvipsnames]{beamer}
\usepackage{beamerthemesplit}
\usepackage{pgfgantt}
\usetheme{Warsaw}
\usecolortheme{}
\usepackage{sidecap}
%\usepackage[french]{babel}
%\usepackage[latin1]{inputenc}
\usepackage[utf8x]{inputenc}
\usetikzlibrary{babel}
\usepackage[T1]{fontenc}
\usepackage{amsfonts}
\usepackage{amsmath,amssymb,theorem}
\usepackage[all]{xy}
\usepackage{textcomp}
%\usepackage{variations}
\usepackage{pstricks,pst-plot,pstricks-add}
\usepackage{bm, listings, ccaption}
\usepackage{xcolor,epic,eepic,multicol}
\graphicspath{{Images_Fichiers/}}

\newcommand{\mat}[4]{ \begin{pmatrix} #1 & #2 \\ #3 & #4 \end{pmatrix} }
\newcommand{\R}{\mathbb{R}}
\newcommand{\gs}[1]{\textbf{\underline{#1}}}
\newcommand{\thm}[2]{\begin{center}
{\color{blue} #1} \\
\fbox{ #2}
\end{center} }

\title{Exercice II.7}
\author{El Houssaini Youness}
\date{\today}
\begin{document}
\maketitle

\section{Sytème de contôle} 
\begin {frame}
On considère le système de contrôle: $$(\Sigma): \quad \left\lbrace\begin{array}{ll}
x'(t) = y(t)+u(t) & ,t\geq 0\\
y'(t) = -y(t)+u(t) & ,t\geq 0\\
|u(.)| \leq 1 
\end{array}\right.$$
où $x$ , $y$ et $u$ sont trois applications: $[0,T] \subset R^{+} \rightarrow R $.

Le but est de joindre en temps minimal la droite x = 0, puis de rester sur cette droite. \\

\end {frame}

\begin{frame}
\frametitle{Question (i) -  Énoncé}
On considère la cible $M_1 = \left\lbrace(0,y),\,|y| ≤ 1\right\rbrace$ et on s’intéresse à l’existence et la caractérisation de trajectoires temps optimales pour le problème $(\Sigma)$ décrit au dessus. \\
\vspace{1cm}
{\color{blue}{Montrer que si un tel contrôle existe, alors on a nécessairement |y(t)| ≤ 1 lorsque
x(t) = 0. Réciproquement, montrer que de tout point (0,y) ∈ $M_1$, part une trajectoire restant dans $M_1$}.}
\end{frame}

\section{Question i}
\begin {frame}
  \frametitle{Question (i) - Montrer que si un tel contrôle existe, alors on a nécessairement |y(t)| ≤ 1 lorsque x(t) = 0.}
En effet, en supposant qu'il y a bien un contôle de ce système: \\
Lorsque $x(t)= 0$, on a $x'(t) = 0$. 
On remplace dans: $$(\Sigma): \quad \left\lbrace\begin{array}{ll}
\fbox{\color{blue}{x'(t) = y(t)+u(t)}} & ,t\geq 0\\
y'(t) = -y(t)+u(t) & ,t\geq 0\\
{|u(.)| \leq 1}
\end{array}\right.$$.

$$(\Sigma) \implies \quad y(t) = -u(t)$$
Or, $|u(.)| \leq 1 $.\\
Alors on a nécessairement: $$|y(t)| ≤ 1 \,\, lorsque \,\, x(t) = 0$$.
 
\end {frame}
\begin{frame}
\frametitle{Question (i) -  Réciproquement, montrer que de tout point (0,y) ∈ $M_1$, part une trajectoire restant dans $M_1$.}
En effet, soit $(0,y) \in M_1$
on remarque que si $x(t)= 0$, alors  $x'(t) = 0$. \\
On remplace dans: $$(\Sigma): \quad \left\lbrace\begin{array}{ll}
\fbox{\color{blue}{x'(t) = y(t)+u(t)}} & ,t\geq 0\\
y'(t) = -y(t)+u(t) & ,t\geq 0\\
{|u(.)| \leq 1}
\end{array}\right.$$.
Et on obtient: $y(t) = -u(t)$ \\
Puis on remplace dans: $$(\Sigma): \quad \left\lbrace\begin{array}{ll}
x'(t) = y(t)+u(t) & ,t\geq 0\\
\fbox{\color{blue}{y'(t) = -y(t)+u(t)}} & ,t\geq 0\\
{|u(.)| \leq 1}
\end{array}\right.$$.
 
\end {frame}
\begin{frame}
Et on en déduit que $\forall t \geq 0$:
$$y(t) = y(0)e^{-2t}$$
On a:  $y \in M_1$ alors:$|y(0)| \leq 1 $ \\
On remarque aussi: $0 \leq e^{-2t} \leq 1$ \\
Donc:  $$\forall t \geq 0 \quad |y(t)| \leq 1$$\\
En d'autre termes: \\
$\forall (0,y) ∈ M_1$, part une trajectoire restant dans $M_1$.
 
\end {frame}

\begin {frame}
\frametitle{Question (ii) -   Etudier l’existence d’une trajectoire temps optimale.}

Le système de contôle $(\Sigma)$ peut être représenté sous la forme canonique linéaire comme suite: $$
\left\lbrace\begin{array}{ll}
X'(t) = AX(t)+Bu(t)& ,t\in [0,T]\\
X(0)=X_0&, X_0\in \R^2\\
\end{array}\right.
$$

En posant $X = (x,y)^T \in \mathbf{M}_{2,1}(\R)$ avec $A=\mat{0}{1}{0}{-1} \in \mathbf{M}_{2,2}(\R)$ et $B=(1,1)^T \in \mathbf{M}_{2,1}(\R)$. Notons que la commande $u \in L^{\infty}(0,T,U)$, où U = [-1,1] est un sous-ensemble mesurable de $\R$.

\end {frame}
\section{Question ii}
\begin{frame}
%\frametitle{Question (ii) -   Etudier l’existence d’une trajectoire temps optimale.}
{Théorème II.1.1. Existence de commandes temps optimales: \\ \color{blue}{On suppose que U est compact. Si le point $x_1$ est accessible depuis $x_0$ avec un contrôle u à valeurs dans U, alors il existe une trajectoire temps-minimale reliant $x_0$ à $x_1$. De plus, $x_1$ est nécessairement extrémal, autrement dit $x_1 \in ∂A(x_0,t^∗)$.}}

{Dans notre cas:}\\

La commande $u \in L^{\infty}(0,T,U)$, où U est un sous-ensemble mesurable de $\R$. en effet,$U =[-1,1]$ est bien compact.\\
On a aussi montré que si $X_0 \in M_1$ alors $\exists X_1$ accessible depuis $X_0$. en plus on a aussi que $X_1 \in M_1$.\\

Donc: il existe une trajectoire temps-minimale reliant $X_0$ à $X_1$.
\end {frame}

\section{Question iii}
\begin{frame}
\frametitle{Question (iii) -   Ènoncé}
On cherche à caractériser les trajectoires temps optimales. On choisit de raisonner
"en temps inverse", en calculant les trajectoires joignant $M_1$ à tout point final.\\

(a) Écrire le Hamiltonien du problème de contrôle optimal et les équations adjointes.
En déduire que le contrôle optimal est bang-bang avec au plus une commutation.\\
(b) En utilisant les conditions de transversalité, montrer que si les conditions initiales sont choisies dans $M_1$, alors le contrôle optimal ne commute pas. Représenter quelques unes des trajectoires associées.
\end {frame}


\begin{frame}
%\frametitle{Question (iii) -   (a) Écrire le Hamiltonien du problème de contrôle optimal et les équations adjointes}
{Théorème II.3.2. Principe du maximum de Pontryagin, version générale \\
\color{blue}{Soit $M_1$ un sous-ensembles de $\R^n$.
Si le contrôle $u ∈ U$ associé à la trajectoire $x(\cdot)$ est optimal sur [0,T], alors il existe une application $p(\cdot) : [0,T] \rightarrow \R^n$ absolument continue appelée vecteur adjoint, et un réel $p^0 \leq 0$, tels que le couple $(p(\cdot),p^0)$ est non trivial, et tels que, pour presque tout $t ∈ [0,T],$\\
$$ x'(t) = \frac {∂H}{∂p}(t,x(t),p(t),p^0,u(t)),$$}}

\end {frame}


\begin{frame}
%\frametitle{Question (iii) -   (a) Écrire le Hamiltonien du problème de contrôle optimal et les équations adjointes}
{\color{blue}{$$p'(t)  = −\frac{∂H}{∂x}(t,x(t),p(t),p^0,u(t)),$$
et on a la condition de maximisation presque partout sur [0,T]
$$H(t,x(t),p(t),p^0,u(t)) = max_{v∈U} H(t,x(t),p(t),p^0 ,v).$$
Si de plus le temps final pour joindre la cible $M_1$ n’est pas fixé, on a la condition au temps final T:
 $$max_{v∈U} H(T,x(T),p(T),p^0 ,v) = −p^0 \frac {∂g}{∂t}(T,x(T)).\quad \quad (*)$$}}

\end {frame}
\begin{frame}
\frametitle{Question (iii) -   (a) Écrire le Hamiltonien du problème de contrôle optimal et les équations adjointes}
Dans notre cas:\\
La cible $M_1 \subset \R^2$ et on définit le Hamiltonien du système de contrôle optimal sur $\R \times \R^2 \times \R^2 \times \R_{-} \times R$ par:

$$H(t,x(t),p(t),p^0 ,u) = <p(t),f(t,x(t),u(t))> + p^0f^0$$
\end {frame}
\begin{frame}
Avec $f : \R × \R^2 × \R \rightarrow \R^2$ est de classe $C^1$, 
$$x'(t) = f(t,x(t),u(t)) = Ax(t)+Bu(t)$$ 
et où les contrôles sont des applications mesurables et bornées définies sur un intervalle $[0,T(u)[$ de $\R_+$ et à valeurs dans $U=[-1,1] ⊂ \R$.

\end {frame}
\begin{frame}
Par ailleurs on définit la fonctionnelle de coût: \\
$$\fbox{$C(T,u) = \int_0^T f^0(s,x(s),u(s))ds + g(T,x(T))$}$$\\

avec $f^0 : \R \times \R^2 \times \R \rightarrow 0$ et $g: \R_{+} \times \R^2 \rightarrow t$\\
Afin d'avoir la reformulation de notre problème de contrôle optimal suivante:
$$inf_{T \in \R^+ \, u \in U} C(T,u) = inf_{T \in \R^+ \, u \in U} T_u$$
\end {frame}

\begin{frame}
\frametitle{Question (iii) -   (a) Écrire le Hamiltonien du problème de contrôle optimal et les équations adjointes}
On applique le Théorème II.3.2:\\

$$ x'(t) = \frac {∂H}{∂p}(t,x(t),p(t),p^0,u(t)) = Ax(t) + Bu(t)$$
%$$p'(t)  = −\frac{∂H}{∂x}(t,x(t),p(t),p^0,u(t)) $$ \\
$$p'(t)  = −\frac{∂<p(t),f(t,x(t),u(t))>}{∂x} = −\frac{∂<p(t),Ax(t)+Bu(t)>}{∂x}$$ \\
$$p'(t)  = −\frac{∂<p(t),Ax(t)>}{∂x}= -A^T p(t)$$ \\
Remarque: Ce qui peut être obtenue par le Théorème II.1.2.
\end {frame}

\begin{frame}
%\frametitle{Question (iii) -   Rappel}
{Théorème II.1.2. Condition d’optimalité et principe du maximum de Pontryagin: \\
\color{blue}{Soit $u ∈ L^{\infty}(0,T,U)$ une commande qui transfère le système ($\Sigma$) de $x(0) = x0$ à $x(T) = xT ∈ \R^2$. Si le temps T est minimum, alors il existe une fonction p non identiquement nulle solution de l’équation adjointe $p'(t) = −A^Tp(t), t ∈ [0,T]$ telle que pour presque tous $s ∈ [0,T], u(s)$ réalise instantanément le maximum de l’Hamiltonien $H : U \ni v→ <p(s), Bv>_{\R^2}$ .}}
Alors le maximum du hamiltonien : $max_{v \in U} <B^Tp(s),v>_{\R^2}$ \\
est atteind en v = 1 ou -1 (bornes de U=[-1,1])
\end {frame}

\begin{frame}
\frametitle{Question (iii) -   (a) Écrire le Hamiltonien du problème de contrôle optimal et les équations adjointes}
$$p'(t)  = -A^T p(t)$$ 
$$\implies p(t) = p(0)e^{-tA^T}$$
En particulier pour t = T:  $\quad p(T) = p(0)e^{-TA^T}$\\
 $$\implies p(0) = p(T)e^{TA^T}$$\\
Le vecteur adjoint est donc:
$$p(t) = p(T)e^{(T-t)A^T}$$

\end {frame}

\begin{frame}
\frametitle{Question (iii) -   (a) En déduire que le contrôle optimal est bang-bang avec au plus une commutation.}
On a la condition de maximisation presque partout sur [0,T]\\
$$H(t,x(t),p(t),p^0,u(t)) = max_{v∈U} H(t,x(t),p(t),p^0,v(t)) $$
$$ \iff u(t) = argmax_{v∈U} <p(t),Ax(t)+Bv(t)> $$
$$ \iff u(t) = argmax_{v∈U} <p(t),Bv(t)> $$
On remarque que le max est atteint dans les bords de U.\\
$u = 1$ si $<p(t),B>_{\R^2} $ positif et $u = -1$ sinon. \\
Donc le contrôle optimal est dit bang-bang avec au plus une commutation.

\end {frame}

\begin{frame}
\frametitle{Question (iii) -   (b) En utilisant les conditions de transversalité, montrer que si les conditions initiales sont choisies dans $M_1$, alors le contrôle optimal ne commute pas.}
On a la condition au temps final a T:\\
$$max_{v∈U} H(T,x(T),p(T),p^0,v) = −p^0 \frac {∂g}{∂t}(T,x(T))$$
$$\iff max_{v∈U} <p(T),Ax(T)+Bv(T)> = -p^0  \frac {∂t}{∂t}(T,x(T)) $$
En posant $p^0 = -1$:\\
$$ max_{v∈U} <p(T),Ax(T)+Bv(T)> = 1$$

\end {frame}

\begin{frame}
\frametitle{Question (iii) -   (b) En utilisant les conditions de transversalité, montrer que si les conditions initiales sont choisies dans $M_1$, alors le contrôle optimal ne commute pas.}
Il faut d'abord déterminer p(T), en utilisant les conditions de transversalité. \\
Pour ce faire:\
\begin{enumerate}
\item Déterminer \underline{l'espace tangent} $\mathbb T_x M_1$;
\item Montrer que $M_1$ est variété de $\R^2$ ayant des espaces tangents en $X(T) ∈ M_1$;
\item Trouver un vecteur adjoint $p(T)$ tel que: $$p(T) \bot \mathbb T_{x(T)} M_1 $$.
\end{enumerate}

\end {frame}

\begin{frame}
\frametitle{Question (iii) -   Déterminer \underline{l'espace tangent} $\mathbb T_x M_1$}
Selon la question (i), $M_1$ peut être aussi représenté comme suite:
$$M_1 = \lbrace X=(x,y) \in \R^2, \quad F(X)=x=0 \rbrace$$
Où F est une fonction de classe $C^1$ de $\R^2$ dans $\R$.\\
Selon la $\underline{Remarque \, II.3.3}$ du cours:\\
$$\mathbb T_xM_1 = \lbrace v ∈ \R^2| ∇F(X).v = 0 \rbrace = \lbrace v= (0,y) ∈ \R^2\rbrace. $$
\end {frame}

\begin{frame}
\frametitle{Question (iii) -   $M_1$ est variété de $\R^2$ ayant des
espaces tangents en $X(T) ∈ M_1$}
\begin{center} 
Définition: 
\end{center}
\vspace{0.2cm}
{\color{blue}{ Soit $M ⊂ \R^n$. On dit que M est une sous-variété de $\R^n$ de dimension p et de classe $C^k$ si pour tout x de M, il existe des voisinages respectifs U de x dans $\R^n$ et V de 0 dans $\R^n$, ainsi qu’un $C^k$ -difféomorphisme $f : U → V$ , envoyant x sur 0 et telle que: $$f(U ∩ M) = V ∩ (\R^p × \lbrace 0 \rbrace).$$}}
On remarque que $f(\R^2 \cap M_1) = \R^2 \cap (\R \times \lbrace 0 \rbrace)$ \\ avec $f(x,y) = (0,y)$.\\
Alors $M_1$ est bien une variété de $\R^2$.
\end {frame}

\begin{frame}
\frametitle{Question (iii) -  Trouver un vecteur $p(T)$ tel que: $$p(T) \bot \mathbb T_{x(T)} M_1. $$}

On a:
$$\mathbb T_xM_1 = \lbrace v ∈ \R^2| ∇F(X).v = 0 \rbrace = \lbrace v= (0,y)\rbrace$$
 $$\implies \, p(T) = \lambda ∇F(X) = \lambda \left(\begin{matrix} 
        1\\ 
        0
      \end{matrix} \right) $$
Avec $\lambda \in \R$, Or, $M_1$ est convexe donc $p(T)$ est unitaire (Voir Remarque 3 du cours commande temps minimum de systèmes linéaires), On prend $\lambda = 1$ pour le reste de l'exercice.

\end {frame}

\begin{frame}
\frametitle{Question (iii) -   (b) En utilisant les conditions de transversalité, montrer que si les conditions initiales sont choisies dans $M_1$, alors le contrôle optimal ne commute pas.}
On a la condition au temps final a T:\\
$$max_{v∈U} H(T,x(T),p(T),p^0,v) = −p^0 \frac {∂g}{∂t}(T,x(T))$$
$$\iff max_{v∈U} <p(T),Ax(T)+Bv(T)> = -p^0  \frac {∂t}{∂t}(T,x(T)) $$
En posant $p^0 = -1$ (choix du cours):\\
$$ max_{v∈U} <p(T),Ax(T)+Bv(T)> = 1 \quad (*)$$

\end {frame}

\begin{frame}
\frametitle{Question (iii) -  (b) En utilisant les conditions de transversalité, montrer que si les conditions initiales sont choisies dans $M_1$, alors le contrôle optimal ne commute pas.}

 $$(*) \, \iff  max_{v∈U} <p(T),Ax(T)+Bv(T)> = 1$$
 
Selon Duhamel:\\
$$AX(T) = Ae^{TA}X_0 + \int_0^T Ae^{(T-s)A}Bu(T)ds$$
$\iff $ $$AX(T) = Ae^{TA}X_0 - \big[e^{(T-s)A}\big]_0^TBu(T)$$
$\iff $ $$ AX(T) = Ae^{TA}X_0 + e^{TA}Bu(T)$$


\end {frame}

\begin{frame}
\frametitle{Question (iii) -  (b) En utilisant les conditions de transversalité, montrer que si les conditions initiales sont choisies dans $M_1$, alors le contrôle optimal ne commute pas.}

On remplace dans (*):
 $$(*) \, \iff   <p(T),Ae^{TA}X_0 + e^{TA}Bu(T) + Bu(T)> = 1$$
 
Avec:\\
$p(T) = \left(\begin{matrix} 
        1\\ 
        0
      \end{matrix} \right)$, $ u(T) = \lbrace -1,1 \rbrace$ et,
$B = \left(\begin{matrix} 
        1\\ 
        1
      \end{matrix} \right)$ et $A = \left(\begin{matrix} 
        0 & 1\\ 
        0 & -1
      \end{matrix} \right)$ et $e^{TA} = \left(\begin{matrix} 
        1 & e^{T} \\ 
        1 & -e^{T}
      \end{matrix} \right)$ et $Ae^{TA} = \left(\begin{matrix} 
        1 & -e^{T} \\ 
        -1 & e^{T}
      \end{matrix} \right)$

\end {frame}

\begin{frame}
\frametitle{Question (iii) -  (b) En utilisant les conditions de transversalité, montrer que si les conditions initiales sont choisies dans $M_1$, alors le contrôle optimal ne commute pas.}

On remplace dans (*):\\
$(*) \, \implies $ 
 $$ <\left(\begin{matrix} 
        1\\ 
        0
      \end{matrix} \right),\left(\begin{matrix} 
        1 & -e^{T} \\ 
        -1 & e^{T}
      \end{matrix} \right)\left(\begin{matrix} 
        x_0\\ 
        y_0
      \end{matrix} \right) +  \big[\left(\begin{matrix} 
        1 & e^{T} \\ 
        1 & -e^{T}
      \end{matrix} \right)\left(\begin{matrix} 
        1\\ 
        1
      \end{matrix} \right)+ \left(\begin{matrix} 
        1\\ 
        1
      \end{matrix} \right)\big]u(T)>_{\R^2} = 1$$
$\implies $ 

$$ \big[ x_0 - y_0e^{T} + (2+e^{T})u(T) \big] = 1 \quad (**)$$
\end {frame}
\begin{frame}
\frametitle{Question (iii) -  (b) En utilisant les conditions de transversalité, montrer que si les conditions initiales sont choisies dans $M_1$, alors le contrôle optimal ne commute pas.}

On suppose maintenant que $X_0 \in M_1$:
$\implies x_0 = 0 et |y_0| \leq 0$\\
Aussi on remplace u(T) par l'une des possibilité du contrôle, soit $u(T)=1$ ou $u(T)=-1$ et
on remplace dans (**) repectivement:\\

$$ \big[ - y_0e^{T} + 2+e^{T} \big] = 1 \, ou \, \big[ - y_0e^{T} - 2-e^{T} \big] = 1$$
$\implies$ 
$$e^{T} = \frac{-1} {1- y_0} \leq 0 \quad ou \quad e^{T} = \frac{-3} {1 + y_0} \leq 0$$
\color{red}{Absurde!}
Donc le contôle ne commute pas dans ce cas de figure. \color{green}{Alors le tram s'arrêtes!}
\end {frame}
\begin{frame}
\frametitle{Question (iii) -  Représenter quelques unes des trajectoires associées.}
Pour rerésenter quelques trajectoires on prend des valeurs de $X_0 \notin M_1.$ On se fixe un temps minimal de $T = ln(2)$ et son contrôle optimale correspondant de $u_t$ donné par la condition de maximisation:
$$ max_{v∈U} <p(t),Bv(t)> $$
$$\iff max_{v∈U} <p(T)e^{(T-t)A^T},Bv(t)> $$
Et Duhamel nous donnes ensuite les trajectoires en fonctions de $X_0$:
$$X(t) = e^{tA}X_0 + \int_0^t e^{(T-s)A}Bu(s)ds$$

\end {frame}

\begin{frame}
\frametitle{Question (iii) -  Représenter quelques unes des trajectoires associées.}
En plus, selon la condition de temps optimal on a les équations suivantes qui nous permettent de tirer les points de départ $X_0$:

$$\left\lbrace\begin{array}{ll}
x_0 =  2y_0 - 3  & ,u_T = 1\\
x_0 =  2y_0 + 5  & ,u_T = -1\\
 
\end{array}\right.$$
\end {frame}

\begin{frame}
\frametitle{Question (iii) -  Représenter quelques unes des trajectoires associées.}
$\forall t \in [0,T]$, ici T =ln(2) pour simplifier les calcules, la procédure se résume comme suite:
\begin{enumerate}
\item Déterminer le contrôle u  $\quad \forall t \in [0,T]$ 
$$u= argmax_{v∈U} <p(T),e^{(T-t)A}Bv(t)>  \quad p(T) = (1,0)^T \in \R^2$$
\item Donner $X_0=(x0,y0) \quad X_0 \notin M_1$ tel que:
$$\left\lbrace\begin{array}{ll}
x_0 =  2y_0 - 3  & ,u_{ln(2)} = 1\\
x_0 =  2y_0 + 5  & ,u_{ln(2)} = -1\\
\end{array}\right.$$
\item En déduire: $X(t) = e^{tA}X_0 + \int_0^t e^{(T-s)A}Bu(s)ds.$
\end{enumerate}

\end {frame}

\end{document}