\documentclass[11pt]{beamer}
\usetheme{PaloAlto}
\usepackage[utf8]{inputenc}
\usepackage{amsmath}
\usepackage{amsfonts}
\usepackage{amssymb}
\author{Anki Lucas \\ El Houssaini Youness}
\title{Exercice II.7}
\setbeamercovered{transparent} 
\setbeamertemplate{navigation symbols}{}  
\institute{Contrôle Optimal} 
\date{20/11/2020} 
%\subject{Gestion d'un corpus de données issues de tests de fatigue sur un banc d'essai} 
\begin{document}

\begin{frame}
\titlepage
\end{frame}

\begin{frame}{Énoncé}
Considérons le système de contrôle 
$$
\left\{
\begin{array}{ll}
x'(t) = y(t)+u(t) & t\geq 0\\
y'(t) = -y(t)+u(t) & t\geq 0\\
|u(.)| \leq 1
\end{array}
\right.
$$
Le but est de joindre en temps minimal la droite $x=0$, puis de rester sur cette droite.
On considère la cible $M_1=\{(0,y)| |y|\leq 1\}$ et on s'intéresse à l'existence et la caractérisation de trajectoires temps optimales pour ce problème.
\end{frame}

\begin{frame}{Question 1}
Montrons que si un tel contrôle existe, alors on a nécessairement $|y(t)|\leq 1$ lorsque $x(t)=0$ :\newline

On suppose qu'un tel contrôle existe. On suppose que $x(t)=0$. Soit $t\geq 0$.\newline

Alors $x'(t)=0$, donc d'après le système de contrôle, $y(t)=-u(t)$.
Or $|u(t)| \leq 1$ donc $|y(t)|\leq 1$.
\end{frame}

\begin{frame}{Question (i)}
Montrons que réciproquement, de tout point $(0,y)\in M_1$ part une trajectoire restant dans $M_1$.\newline

En effet, soit $(0,y) \in M_1$, on remarque que si $x(t)=0$, alors $x'(t)=0$.\newline
On déduit donc de la première équation que $y(t)=-u(t)$.\newline
En injectant dans la deuxième, on trouve que $y'(t)=-2y(t)$.\newline

Par conséquent, pour tout $t\geq 0$ :
\begin{equation}
y(t)=y(0)e^{-2t}
\end{equation}

C'est une trajectoire solution du problème en postulant que x(t)=0. Qui est donc l'unique solution par  le théorème de Cauchy-Lipschitz.\\ (Théorème I.2.1. Cauchy-Lipschitz du polycopié).  
\end{frame}

\begin{frame}{Question (i)}
En effet: \\
\vspace{0.2cm} Comme $y \in M_1$, alors $|y(0)| \leq 1$.\newline
De plus $0 \leq e^{-2t} \leq 1$.\newline

Donc pour tout $t \geq 0$ : 
\begin{equation*}
|y(t)| \leq 1
\end{equation*}

En d'autres termes, pour tout $(0,y)$ dans $M_1$, part une trajectoire restant dans $M_1$.

\end{frame}

\begin{frame}{Question (ii)}
Étudions l'existence d'une trajectoire en temps optimale.\newline

Posons $X = \begin{pmatrix} x \\ y \end{pmatrix}$, 
$X' = \begin{pmatrix} x' \\ y' \end{pmatrix}$.
Alors le système de contrôle peut se mettre sous la forme :
\begin{equation}
X'= \begin{pmatrix} 0 & 1 \\ 0 & -1 \end{pmatrix} \begin{pmatrix} x \\ y \end{pmatrix}
+ \begin{pmatrix} 1 \\ 1 \end{pmatrix} u
\end{equation} 
Nous avons donc la matrice de Kalman associée au problème 
\begin{equation}
Kal[B|AB]= \begin{pmatrix} 1 & 1 \\ 1 & -1 \end{pmatrix}
\end{equation}
qui est de rang $2$, donc il existe une trajectoire en temps optimale.
\end{frame}

\begin{frame}{Question (iii)(a)}
On cherche à caractériser les trajectoires temps optimales. On choisit de raisonner "en temps inverse", en calculant les trajectoires joignant $M_1$ à tout point final.\newline

Pour cela, on pose $\tilde{X} = X(T-t)$, par facilité de notation, on continuera d'écrire $X$ au lieu de $\tilde{X}$.\newline

Montrons, en écrivant le Hamiltonien et les équations adjointes, que le contrôle optimale est bang-bang avec au plus une commutation.
\end{frame}

\begin{frame}{Question (iii)(a)}
On écrit notre système sous la forme :
\begin{equation}
X'(t) = f(t,X(t),u(t)
\end{equation}
avec 
\begin{align*}
f : \mathbb{R}\times\mathbb{R}^2\times\mathbb{R} &\longrightarrow \mathbb{R}^2 \\
(t,X,u) &\longmapsto \begin{pmatrix} -y(t)-u(t) \\ y(t)-u(t) \end{pmatrix}
\end{align*}

On veut minimiser 
\begin{equation}
C(T,u)=\int_0^T f^0(s,X(s),u(s))ds + g(T,X(T))
\end{equation}
Ici $f^0 = 0$ et $g(T,X(T))=T$.
\end{frame}

\begin{frame}{Question (iii)(a)}
On rappelle l'expression du Hamiltonien : 
\begin{align*}
H : &\mathbb{R}\times\mathbb{R}^2\times\mathbb{R}^2\times\mathbb{R}\times\mathbb{R}  \longrightarrow \mathbb{R} \\
&(t,X,P,p^0,u) \longmapsto <P,f(t,X(t),u(t)> + p^0 f^0(t,X(t),u(t))
\end{align*}

Soit $P = \begin{pmatrix} p_1 \\ p_2 \end{pmatrix}$. On a donc 
\begin{equation}
H = -p_1(y+u) + p_2(y-u)
\end{equation}
\end{frame}

\begin{frame}{Question (iii)(a)}
On applique le PMP. Soit $u$ optimal, il existe $P$ absolument continu, $p^0$ négatif tel que :
\begin{align}
&X'(t) = \frac{\partial H}{\partial P} = \begin{pmatrix} -y(t)-u(t) \\ y(t)-u(t) \end{pmatrix} \\
&P'(t) = - \frac{\partial H}{\partial X} = \begin{pmatrix} 0 \\ p_1(t)-p_2(t) \end{pmatrix} = \begin{pmatrix} p_1'(t) \\ p_2'(t) \end{pmatrix}
\end{align}
\end{frame}

\begin{frame}{Question (iii)(a)}
On a donc 
\begin{align*}
H(t,X,P,p^0,u) &= \max_{v\in[-1,1]} H(t,X,P,p^0,v) \\
&= \max_{v\in[-1,1]} -p_1(y+v) + p_2(y-v) \\
&= (p_2-p_1)y - \min_{v\in[-1,1]} v(p_1+p_2)
\end{align*}

Donc $u(t)=-signe(p_1(t)+p_2(t))$.

\end{frame}

\begin{frame}{Question (iii)(a)}

On sait que $P= \begin{pmatrix} p_1 \\ p_2 \end{pmatrix}$ résout le système :

$$
\left\{
\begin{array}{ll}
p_1' = 0 \\
p_2' = p_1-p_2
\end{array}
\right.
$$

Ce système a pour solution :
$$
\left\{
\begin{array}{ll}
p_1(t) = c_1 \\
p_2(t) = c_1 + c_2e^{-t}
\end{array}
\right.
$$
Avec $c_1,c_2$ deux constantes.\newline
En évaluant $p_2$ en $0$, on trouve que  
\begin{equation}
p_2(t) = c_1 + (p_2(0)-c_1)e^{-t}
\end{equation}
\end{frame}

\begin{frame}{Question (iii)(a)}
Donc 
\begin{equation}
H = (p_1-p_2)y - \min_{v\in[-1,1]} v(2c_1 + (p_2(0)-c_1)e^{-t})
\end{equation}

On constate que la fonction $t\longmapsto 2c_1 + (p_2(0)-c_1)e^{-t}$ est strictement monotone, on en déduit donc que $u$ est bang-bang.
De plus il n'y a qu'un seul changement de signe au plus, donc il y a au plus un point de commutation .
\end{frame}

\begin{frame}{Question (iii)(b)}
Montrons, en utilisant les conditions de transversalité, que si les conditions initiales sont choisies dans $M_1$, alors le contrôle optimal ne commute pas .\newline

On suppose que les conditions initiales sont choisies dans $M_1$.\
On rappelle que  $M_1 = \{(0,y)| |y|\leq 1\}$
On a donc les espaces tangents $T_{X(0)}M_1 = {0}\times\mathbb{R}$.\newline

La condition de transversalité sur $M_1$ nous donne 
\begin{equation*}
P(0) \perp T_{X(0)}M_1
\end{equation*}
\end{frame}

\begin{frame}{Question (iii)(b)}
On obtient donc les conditions suivantes:

$$
\left\{
\begin{array}{ll}
p_1(0) = \pm 1 \\
p_2(0) = 0 \\
\end{array}
\right.
$$

En nous rappelant que 
$$
\left\{
\begin{array}{ll}
p_1(t) = c_1 \\
p_2(t) = c_1 + (p_2(0)-c_1)e^{-t}
\end{array}
\right.
$$

on en déduit que $c_1=\pm 1$, donc que: $$p_1(t) + p_2(t) = \pm (2-e^{-t}) \neq 0 \quad \forall t \geq 0$$\\Or, $u(t)=-signe(p_1(t)+p_2(t))$, donc $u$ ne commute pas.
\end{frame}

\begin{frame}{Question (iii)(b)}
Par exemple, si $c_1 = 1$, on a $u(t)=-1$.\newline
En injectant dans le système de contrôle on a $(x,y)$ solution de:

$$
\left\{
\begin{array}{ll}
x'(t) = -y(t)+1 \\
y'(t) = y(t)+1
\end{array}
\right.
$$

On déduit directement que:
$$
\left\{
\begin{array}{ll}
x(t)= -y_1e^t  + 2t+ x_1 \\
y(t)=y_1e^t-1
\end{array}
\right.
$$
Avec:
$$
\left\{
\begin{array}{ll}
x_1= x(0)+y(0) +1 \\
y_1=y(0)+1
\end{array}
\right.
$$
En partant du point $x=x(0)=0,y=y(0)=1$, on en déduit que
$$
\left\{
\begin{array}{ll}
x(t) &= -2e^t +2 t+2 \\
y(t) &= 2e^t -1
\end{array} 
\right.
$$
\end{frame}

\end{document}